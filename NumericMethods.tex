\documentclass[10pt]{article}
\input{styles/AESh.sty}
\DeclareMathOperator{\Arctg}{Arctg}
\makeatletter
\renewcommand\subsection{\@startsection {subsection}{1}{\z@}%
	{-2ex \@plus -1ex \@minus -.5ex}%
	{.3ex \@plus.2ex \@minus -.1ex}%
	{\centering\normalfont\large\bfseries}}

\renewcommand\section{\@startsection {section}{1}{\z@}%
	{-3.5ex \@plus -1ex \@minus -.2ex}%
	{2.3ex \@plus.2ex}%
	{\centering\normalfont\Large\bfseries\textsc}}
\renewcommand\subsubsection{\@startsection {subsubsection}{1}{\z@}%
	{-3.5ex \@plus -1ex \@minus -.2ex}%
	{2.3ex \@plus.2ex}%
	{\centering\normalfont\normalsize\bfseries}}
\makeatother
\newcommand{\indexStyle}{\textstyle}
\makeatletter
\renewcommand\part{%
	\if@noskipsec \leavevmode \fi
	\par
	\addvspace{4ex}%
	\@afterindentfalse
	\secdef\@part\@spart}
\def\@part[#1]#2{%
	\ifnum \c@secnumdepth >\m@ne
	\refstepcounter{part}%
	\addcontentsline{toc}{part}{\thepart\hspace{1em}#1}%
	\else
	\addcontentsline{toc}{part}{#1}%
	\fi
	{\parindent \z@ \raggedright
		\interlinepenalty \@M
		\normalfont
		\ifnum \c@secnumdepth >\m@ne
		\centering\Large\bfseries \partname\nobreakspace\thepart
		\par\nobreak
		\fi
		\huge \bfseries #2%
		\markboth{}{}\par}%
	\nobreak
	\vskip 3ex
	\@afterheading}
\def\@spart#1{%
	{\parindent \z@ \raggedright
		\interlinepenalty \@M
		\normalfont
		\huge \bfseries #1\par}%
	\nobreak
	\vskip 3ex
	\@afterheading}
\makeatother 

\makeatletter
\AtBeginDocument{\renewcommand{\tableofcontents}{
		\@cfttocstart
		\par
		\begingroup
		\parindent\z@ \parskip\cftparskip\centering
		\@cftmaketoctitle
		\if@cfttocbibind
		\@cftdobibtoc
		\fi
		\@starttoc{toc}%
		\endgroup
		\@cfttocfinish}
	\makeatother}

\begin{document}
	\setlength{\abovedisplayskip}{3pt plus 3pt minus 2pt}
	\setlength{\abovedisplayshortskip}{3pt plus 2pt minus 3pt}
	\setlength{\belowdisplayskip}{3pt plus 3pt minus 2pt}
	\setlength{\belowdisplayshortskip}{3pt plus 2pt minus 3pt}
	\setlength{\textfloatsep}{1em plus .4em minus .3em}
	\setlength{\abovecaptionskip}{0.5em plus .4em minus .1em}
	\setlength{\belowcaptionskip}{0.5em plus .4em minus .1em}
	\begin{titlepage}
		\thispagestyle{empty}
		\begin{center}
			\LARGE{\textsc{Skolkovo Institute of Science and Technology}}\\
			
			\normalsize Center of Material Technologies\\
			\vspace{2cm}
			\Large
			Report on the course\\
			\LARGE{\textbf{«Numerical Methods in Engineering and Science».}}\\[.4em]
			\vspace{10cm}
			\normalsize
			\begin{flushright}
				\begin{tabular}{rl}
					\textbf{Executor:}\\
					PhD Student A.S.~Kulikov 
				\end{tabular}
			\end{flushright}
			\vspace{0.5cm}
			\begin{flushright}
				\begin{tabular}{rl}
					\\
					\textbf{Lecturer:}\\
					Prof. O.V.~Vasilyev
				\end{tabular}
			\end{flushright}
			\vfill
			{Skolkovo\\
				2023}
		\end{center}
	\end{titlepage}
	\setcounter{page}{2}
	
	\tableofcontents
	\newpage
	
	\part{Lagrange and Hermite interpolation}
	
	\par Lagrange and Hermite interpolants are considered for a set of functions and grid distributions in the interval [0, 1]. \\
	Functions:
	\begin{enumerate}
		\item $ \frac{1}{1+x^2} $.
		\item $ (x-\frac{1}{2})^2 sign(x-\frac{1}{2}) $.
		\item $ |x-\frac{1}{2}| $.
		\item $ \sqrt{1-x^2} $.
	\end{enumerate}

	Corresponding derivatives:
	\begin{enumerate}
		\item $ \frac{-2x}{(1+x^2)^2} $.
		\item $ 2(x-\frac{1}{2}) sign(x-\frac{1}{2}) $.
		\item $ sign(x-\frac{1}{2}) $.
		\item $ \frac{-x}{\sqrt{1-x^2}} $.
	\end{enumerate}

	Grid distributions:
	\begin{enumerate}
		\item Equispaced: $ x_i=\frac{i}{N}, \quad i=0, ..., N $.
		\item Chebyshev: $ \frac{1}{2} - \frac{1}{2}cos(\frac{i}{N}\pi), \quad i=0, ..., N  $.
		\item Asin: $ \frac{1}{2} + \frac{1}{\pi}sin^{-1}\left(\frac{2i}{N}-1\right), \quad i=0, ..., N  $.
	\end{enumerate}

where $N$ is the number of data points. The error of Lagrange interpolant is majorated by $\dfrac{\|y^{N+1}\|\|F_N\|}{(N+1)!}$ and of Hermit by $\dfrac{\|y^{2N+2}\|\|F_N^2\|}{(2N+2)!}$, where $F_N(x)=\prod_{j=0}^{N}(x-x_j)$. Since $\|F_N\|$ is minimal on Chebyshev nodes we expect our interpolation to be the most accurate on them.
	
	\section{$ \frac{1}{1+x^2}  $}
	\subsection{Lagrange interpolant}
	\begin{figure}[H]
		\begin{minipage}[h]{0.34\linewidth}
			\center{\includegraphics[width=1.0\linewidth]{LagrangeHermit/Lagr_Equi_F1_N10.png}}
		\end{minipage}%
		%\hfill
		\begin{minipage}[h]{0.34\linewidth}
			\center{\includegraphics[width=1.0\linewidth]{LagrangeHermit/Lagr_Cheb_F1_N10.png}}
		\end{minipage}%
		%\hfill
		\begin{minipage}[h]{0.34\linewidth}
			\center{\includegraphics[width=1.0\linewidth]{LagrangeHermit/Lagr_Asin_F1_N10.png}}
		\end{minipage}%
		\vfill
		\begin{minipage}[h]{0.34\linewidth}
			\center{\includegraphics[width=1.0\linewidth]{LagrangeHermit/Lagr_Equi_F1_N20.png}}
		\end{minipage}%
		%\hfill
		\begin{minipage}[h]{0.34\linewidth}
			\center{\includegraphics[width=1.0\linewidth]{LagrangeHermit/Lagr_Cheb_F1_N20.png}} 
		\end{minipage}%
		%\hfill
		\begin{minipage}[h]{0.34\linewidth}
			\center{\includegraphics[width=1.0\linewidth]{LagrangeHermit/Lagr_Asin_F1_N20.png}} 
		\end{minipage}%
		\vfill
		\begin{minipage}[h]{0.34\linewidth}
			\center{\includegraphics[width=1.0\linewidth]{LagrangeHermit/Lagr_Equi_F1_N40.png}} 
		\end{minipage}%
		%\hfill
		\begin{minipage}[h]{0.34\linewidth}
			\center{\includegraphics[width=1.0\linewidth]{LagrangeHermit/Lagr_Cheb_F1_N40.png}} 
		\end{minipage}%
		%\hfill
		\begin{minipage}[h]{0.34\linewidth}
			\center{\includegraphics[width=1.0\linewidth]{LagrangeHermit/Lagr_Asin_F1_N40.png}} 
		\end{minipage}%
		\vfill
		\begin{minipage}[h]{0.34\linewidth}
			\center{\includegraphics[width=1.0\linewidth]{LagrangeHermit/Lagr_Equi_F1_N80.png}} 
		\end{minipage}%
		%\hfill
		\begin{minipage}[h]{0.34\linewidth}
			\center{\includegraphics[width=1.0\linewidth]{LagrangeHermit/Lagr_Cheb_F1_N80.png}} 
		\end{minipage}%
		%\hfill
		\begin{minipage}[h]{0.34\linewidth}
			\center{\includegraphics[width=1.0\linewidth]{LagrangeHermit/Lagr_Asin_F1_N80.png}} 
		\end{minipage}%
		\caption{Results of Lagrange interpolation for 10, 20, 40 and 80 data points. The function is pictured with blue, its interpolant with red. First colomn corresponds to Equispaced data point distribution, second to Chebyshev and third to Asin.}
		%\label{ris:Area1-5}
	\end{figure}
\newpage
	\subsection{Hermit interpolant}
		\begin{figure}[H]
		\begin{minipage}[h]{0.34\linewidth}
			\center{\includegraphics[width=1.0\linewidth]{LagrangeHermit/Herm_Equi_F1_N5.png}}
		\end{minipage}%
		%\hfill
		\begin{minipage}[h]{0.34\linewidth}
			\center{\includegraphics[width=1.0\linewidth]{LagrangeHermit/Herm_Cheb_F1_N5.png}}
		\end{minipage}%
		%\hfill
		\begin{minipage}[h]{0.34\linewidth}
			\center{\includegraphics[width=1.0\linewidth]{LagrangeHermit/Herm_Asin_F1_N5.png}}
		\end{minipage}%
		\vfill
		\begin{minipage}[h]{0.34\linewidth}
			\center{\includegraphics[width=1.0\linewidth]{LagrangeHermit/Herm_Equi_F1_N10.png}}
		\end{minipage}%
		%\hfill
		\begin{minipage}[h]{0.34\linewidth}
			\center{\includegraphics[width=1.0\linewidth]{LagrangeHermit/Herm_Cheb_F1_N10.png}} 
		\end{minipage}%
		%\hfill
		\begin{minipage}[h]{0.34\linewidth}
			\center{\includegraphics[width=1.0\linewidth]{LagrangeHermit/Herm_Asin_F1_N10.png}} 
		\end{minipage}%
		\vfill
		\begin{minipage}[h]{0.34\linewidth}
			\center{\includegraphics[width=1.0\linewidth]{LagrangeHermit/Herm_Equi_F1_N20.png}} 
		\end{minipage}%
		%\hfill
		\begin{minipage}[h]{0.34\linewidth}
			\center{\includegraphics[width=1.0\linewidth]{LagrangeHermit/Herm_Cheb_F1_N20.png}} 
		\end{minipage}%
		%\hfill
		\begin{minipage}[h]{0.34\linewidth}
			\center{\includegraphics[width=1.0\linewidth]{LagrangeHermit/Herm_Asin_F1_N20.png}} 
		\end{minipage}%
		\vfill
		\begin{minipage}[h]{0.34\linewidth}
			\center{\includegraphics[width=1.0\linewidth]{LagrangeHermit/Herm_Equi_F1_N40.png}} 
		\end{minipage}%
		%\hfill
		\begin{minipage}[h]{0.34\linewidth}
			\center{\includegraphics[width=1.0\linewidth]{LagrangeHermit/Herm_Cheb_F1_N40.png}} 
		\end{minipage}%
		%\hfill
		\begin{minipage}[h]{0.34\linewidth}
			\center{\includegraphics[width=1.0\linewidth]{LagrangeHermit/Herm_Asin_F1_N40.png}} 
		\end{minipage}%
		\caption{Results of Hermit interpolation for 5, 10, 20 and 40 data points. The function is pictured with blue, its interpolant with red. First colomn corresponds to Equispaced data point distribution, second to Chebyshev and third to Asin.}
		%\label{ris:Area1-5}
	\end{figure}
	\newpage
	\subsection{Accuracy analysis}
		\begin{figure}[H]
		\begin{minipage}[h]{0.5\linewidth}
			\center{\includegraphics[width=1.0\linewidth]{LagrangeHermit/Lagr_eq_err_F1.png}}
			\caption{Dependence of error on the number of data\\ points for Lagrange interpolant and Equispaced point distribution.}
		\end{minipage}%
		\hspace{0.5cm}
		%\hfill
		\begin{minipage}[h]{0.5\linewidth}
			\center{\includegraphics[width=1.0\linewidth]{LagrangeHermit/Herm_eq_err_F1.png}} 
			\caption{Dependence of error on the number of data\\ points for Hermit interpolant and Equispaced point distribution.}
		\end{minipage}%
		\vfill
		\begin{minipage}[h]{0.5\linewidth}
			\center{\includegraphics[width=1.0\linewidth]{LagrangeHermit/Lagr_cheb_err_F1.png}} 
			\caption{Dependence of error on the number of data\\ points for Lagrange interpolant and Chebyshev point distribution.}
		\end{minipage}%
	\hspace{0.5cm}
		%\hfill
		\begin{minipage}[h]{0.5\linewidth}
			\center{\includegraphics[width=1.0\linewidth]{LagrangeHermit/Herm_cheb_err_F1.png}} 
				\caption{Dependence of error on the number of data\\ points for Hermit interpolant and Chebyshev point distribution.}
		\end{minipage}%
		%\hfill
		\vfill
		\begin{minipage}[h]{0.5\linewidth}
			\center{\includegraphics[width=1.0\linewidth]{LagrangeHermit/Lagr_asin_err_F1.png}} 
				\caption{Dependence of error on the number of data\\ points for Lagrange interpolant and Asin point distribution.}
		\end{minipage}%
	\hspace{0.5cm}
		%\hfill
		\begin{minipage}[h]{0.5\linewidth}
			\center{\includegraphics[width=1.0\linewidth]{LagrangeHermit/Herm_asin_err_F1.png}} 
			\caption{Dependence of error on the number of data\\ points for Hermit interpolant and Asin point distribution.}
		\end{minipage}%
		%\hfill
	\end{figure}

As expected, Chebyshev point distribution provides the best accuracy. Second best is for Equispaced distribution and the worst is for Asin. Since Chebyshev point distribution is the most accurate and it gets denser the closer the interval boundaries are, it seems logical that Asin is the least accurate, because it gets rarer when approaching them. For this particular function Hermit interpolation turned out to be more accurate than Lagrange. Most likely it is due to the function smoothness. Note that for high $N$ the error grows rapidly, which is not only connected with $\|y^{N+1}\|$ growth, but also with "round-off" error accumulation.  
	\newpage
	
	%%%%%%%%%%%%%%%2
		\section{$ (x-\frac{1}{2})^2 sign(x-\frac{1}{2}) $}
	\subsection{Lagrange interpolant}
	\begin{figure}[H]
		\begin{minipage}[h]{0.34\linewidth}
			\center{\includegraphics[width=1.0\linewidth]{LagrangeHermit/Lagr_Equi_F2_N10.png}}
		\end{minipage}%
		%\hfill
		\begin{minipage}[h]{0.34\linewidth}
			\center{\includegraphics[width=1.0\linewidth]{LagrangeHermit/Lagr_Cheb_F2_N10.png}}
		\end{minipage}%
		%\hfill
		\begin{minipage}[h]{0.34\linewidth}
			\center{\includegraphics[width=1.0\linewidth]{LagrangeHermit/Lagr_Asin_F2_N10.png}}
		\end{minipage}%
		\vfill
		\begin{minipage}[h]{0.34\linewidth}
			\center{\includegraphics[width=1.0\linewidth]{LagrangeHermit/Lagr_Equi_F2_N20.png}}
		\end{minipage}%
		%\hfill
		\begin{minipage}[h]{0.34\linewidth}
			\center{\includegraphics[width=1.0\linewidth]{LagrangeHermit/Lagr_Cheb_F2_N20.png}} 
		\end{minipage}%
		%\hfill
		\begin{minipage}[h]{0.34\linewidth}
			\center{\includegraphics[width=1.0\linewidth]{LagrangeHermit/Lagr_Asin_F2_N20.png}} 
		\end{minipage}%
		\vfill
		\begin{minipage}[h]{0.34\linewidth}
			\center{\includegraphics[width=1.0\linewidth]{LagrangeHermit/Lagr_Equi_F2_N40.png}} 
		\end{minipage}%
		%\hfill
		\begin{minipage}[h]{0.34\linewidth}
			\center{\includegraphics[width=1.0\linewidth]{LagrangeHermit/Lagr_Cheb_F2_N40.png}} 
		\end{minipage}%
		%\hfill
		\begin{minipage}[h]{0.34\linewidth}
			\center{\includegraphics[width=1.0\linewidth]{LagrangeHermit/Lagr_Asin_F2_N40.png}} 
		\end{minipage}%
		\vfill
		\begin{minipage}[h]{0.34\linewidth}
			\center{\includegraphics[width=1.0\linewidth]{LagrangeHermit/Lagr_Equi_F2_N80.png}} 
		\end{minipage}%
		%\hfill
		\begin{minipage}[h]{0.34\linewidth}
			\center{\includegraphics[width=1.0\linewidth]{LagrangeHermit/Lagr_Cheb_F2_N80.png}} 
		\end{minipage}%
		%\hfill
		\begin{minipage}[h]{0.34\linewidth}
			\center{\includegraphics[width=1.0\linewidth]{LagrangeHermit/Lagr_Asin_F2_N80.png}} 
		\end{minipage}%
		\caption{Results of Lagrange interpolation for 10, 20, 40 and 80 data points. The function is pictured with blue, its interpolant with red. First colomn corresponds to Equispaced data point distribution, second to Chebyshev and third to Asin.}
		%\label{ris:Area1-5}
	\end{figure}
	\newpage
	\subsection{Hermit interpolant}
	\begin{figure}[H]
		\begin{minipage}[h]{0.34\linewidth}
			\center{\includegraphics[width=1.0\linewidth]{LagrangeHermit/Herm_Equi_F2_N5.png}}
		\end{minipage}%
		%\hfill
		\begin{minipage}[h]{0.34\linewidth}
			\center{\includegraphics[width=1.0\linewidth]{LagrangeHermit/Herm_Cheb_F2_N5.png}}
		\end{minipage}%
		%\hfill
		\begin{minipage}[h]{0.34\linewidth}
			\center{\includegraphics[width=1.0\linewidth]{LagrangeHermit/Herm_Asin_F2_N5.png}}
		\end{minipage}%
		\vfill
		\begin{minipage}[h]{0.34\linewidth}
			\center{\includegraphics[width=1.0\linewidth]{LagrangeHermit/Herm_Equi_F2_N10.png}}
		\end{minipage}%
		%\hfill
		\begin{minipage}[h]{0.34\linewidth}
			\center{\includegraphics[width=1.0\linewidth]{LagrangeHermit/Herm_Cheb_F2_N10.png}} 
		\end{minipage}%
		%\hfill
		\begin{minipage}[h]{0.34\linewidth}
			\center{\includegraphics[width=1.0\linewidth]{LagrangeHermit/Herm_Asin_F2_N10.png}} 
		\end{minipage}%
		\vfill
		\begin{minipage}[h]{0.34\linewidth}
			\center{\includegraphics[width=1.0\linewidth]{LagrangeHermit/Herm_Equi_F2_N20.png}} 
		\end{minipage}%
		%\hfill
		\begin{minipage}[h]{0.34\linewidth}
			\center{\includegraphics[width=1.0\linewidth]{LagrangeHermit/Herm_Cheb_F2_N20.png}} 
		\end{minipage}%
		%\hfill
		\begin{minipage}[h]{0.34\linewidth}
			\center{\includegraphics[width=1.0\linewidth]{LagrangeHermit/Herm_Asin_F2_N20.png}} 
		\end{minipage}%
		\vfill
		\begin{minipage}[h]{0.34\linewidth}
			\center{\includegraphics[width=1.0\linewidth]{LagrangeHermit/Herm_Equi_F2_N40.png}} 
		\end{minipage}%
		%\hfill
		\begin{minipage}[h]{0.34\linewidth}
			\center{\includegraphics[width=1.0\linewidth]{LagrangeHermit/Herm_Cheb_F2_N40.png}} 
		\end{minipage}%
		%\hfill
		\begin{minipage}[h]{0.34\linewidth}
			\center{\includegraphics[width=1.0\linewidth]{LagrangeHermit/Herm_Asin_F2_N40.png}} 
		\end{minipage}%
		\caption{Results of Hermit interpolation for 5, 10, 20 and 40 data points. The function is pictured with blue, its interpolant with red. First colomn corresponds to Equispaced data point distribution, second to Chebyshev and third to Asin.}
		%\label{ris:Area1-5}
	\end{figure}
	\newpage
	\subsection{Accuracy analysis}
	\begin{figure}[H]
		\begin{minipage}[h]{0.45\linewidth}
			\center{\includegraphics[width=1.0\linewidth]{LagrangeHermit/Lagr_eq_err_F2.png}}
			\caption{Dependence of error on the number of data\\ points for Lagrange interpolant and Equispaced point distribution.}
		\end{minipage}%
		\hspace{0.5cm}
		%\hfill
		\begin{minipage}[h]{0.45\linewidth}
			\center{\includegraphics[width=1.0\linewidth]{LagrangeHermit/Herm_eq_err_F2.png}} 
			\caption{Dependence of error on the number of data\\ points for Hermit interpolant and Equispaced point distribution.}
		\end{minipage}%
		\vfill
		\begin{minipage}[h]{0.45\linewidth}
			\center{\includegraphics[width=1.0\linewidth]{LagrangeHermit/Lagr_cheb_err_F2.png}} 
			\caption{Dependence of error on the number of data\\ points for Lagrange interpolant and Chebyshev point distribution.}
		\end{minipage}%
		\hspace{0.5cm}
		%\hfill
		\begin{minipage}[h]{0.45\linewidth}
			\center{\includegraphics[width=1.0\linewidth]{LagrangeHermit/Herm_cheb_err_F2.png}} 
			\caption{Dependence of error on the number of data\\ points for Hermit interpolant and Chebyshev point distribution.}
		\end{minipage}%
		%\hfill
		\vfill
		\begin{minipage}[h]{0.45\linewidth}
			\center{\includegraphics[width=1.0\linewidth]{LagrangeHermit/Lagr_asin_err_F2.png}} 
			\caption{Dependence of error on the number of data\\ points for Lagrange interpolant and Asin point distribution.}
		\end{minipage}%
		\hspace{0.5cm}
		%\hfill
		\begin{minipage}[h]{0.45\linewidth}
			\center{\includegraphics[width=1.0\linewidth]{LagrangeHermit/Herm_asin_err_F2.png}} 
			\caption{Dependence of error on the number of data\\ points for Hermit interpolant and Asin point distribution.}
		\end{minipage}%
		%\hfill
	\end{figure}
The accuracy for this function interpolation is significantly smaller. This is because this function is not as smooth as the previous one. For the same reason Lagrange interpolant is more accurate than Hermit.
	\newpage
	
	%%%%%%%%%%%%%%%%%%%3
		\section{$ |x-\frac{1}{2}|  $}
	\subsection{Lagrange interpolant}
	\begin{figure}[H]
		\begin{minipage}[h]{0.34\linewidth}
			\center{\includegraphics[width=1.0\linewidth]{LagrangeHermit/Lagr_Equi_F3_N10.png}}
		\end{minipage}%
		%\hfill
		\begin{minipage}[h]{0.34\linewidth}
			\center{\includegraphics[width=1.0\linewidth]{LagrangeHermit/Lagr_Cheb_F3_N10.png}}
		\end{minipage}%
		%\hfill
		\begin{minipage}[h]{0.34\linewidth}
			\center{\includegraphics[width=1.0\linewidth]{LagrangeHermit/Lagr_Asin_F3_N10.png}}
		\end{minipage}%
		\vfill
		\begin{minipage}[h]{0.34\linewidth}
			\center{\includegraphics[width=1.0\linewidth]{LagrangeHermit/Lagr_Equi_F3_N20.png}}
		\end{minipage}%
		%\hfill
		\begin{minipage}[h]{0.34\linewidth}
			\center{\includegraphics[width=1.0\linewidth]{LagrangeHermit/Lagr_Cheb_F3_N20.png}} 
		\end{minipage}%
		%\hfill
		\begin{minipage}[h]{0.34\linewidth}
			\center{\includegraphics[width=1.0\linewidth]{LagrangeHermit/Lagr_Asin_F3_N20.png}} 
		\end{minipage}%
		\vfill
		\begin{minipage}[h]{0.34\linewidth}
			\center{\includegraphics[width=1.0\linewidth]{LagrangeHermit/Lagr_Equi_F3_N40.png}} 
		\end{minipage}%
		%\hfill
		\begin{minipage}[h]{0.34\linewidth}
			\center{\includegraphics[width=1.0\linewidth]{LagrangeHermit/Lagr_Cheb_F3_N40.png}} 
		\end{minipage}%
		%\hfill
		\begin{minipage}[h]{0.34\linewidth}
			\center{\includegraphics[width=1.0\linewidth]{LagrangeHermit/Lagr_Asin_F3_N40.png}} 
		\end{minipage}%
		\vfill
		\begin{minipage}[h]{0.34\linewidth}
			\center{\includegraphics[width=1.0\linewidth]{LagrangeHermit/Lagr_Equi_F3_N80.png}} 
		\end{minipage}%
		%\hfill
		\begin{minipage}[h]{0.34\linewidth}
			\center{\includegraphics[width=1.0\linewidth]{LagrangeHermit/Lagr_Cheb_F3_N80.png}} 
		\end{minipage}%
		%\hfill
		\begin{minipage}[h]{0.34\linewidth}
			\center{\includegraphics[width=1.0\linewidth]{LagrangeHermit/Lagr_Asin_F3_N80.png}} 
		\end{minipage}%
		\caption{Results of Lagrange interpolation for 10, 20, 40 and 80 data points. The function is pictured with blue, its interpolant with red. First colomn corresponds to Equispaced data point distribution, second to Chebyshev and third to Asin.}
		%\label{ris:Area1-5}
	\end{figure}
	\newpage
	\subsection{Hermit interpolant}
	\begin{figure}[H]
		\begin{minipage}[h]{0.34\linewidth}
			\center{\includegraphics[width=1.0\linewidth]{LagrangeHermit/Herm_Equi_F3_N5.png}}
		\end{minipage}%
		%\hfill
		\begin{minipage}[h]{0.34\linewidth}
			\center{\includegraphics[width=1.0\linewidth]{LagrangeHermit/Herm_Cheb_F3_N5.png}}
		\end{minipage}%
		%\hfill
		\begin{minipage}[h]{0.34\linewidth}
			\center{\includegraphics[width=1.0\linewidth]{LagrangeHermit/Herm_Asin_F3_N5.png}}
		\end{minipage}%
		\vfill
		\begin{minipage}[h]{0.34\linewidth}
			\center{\includegraphics[width=1.0\linewidth]{LagrangeHermit/Herm_Equi_F3_N10.png}}
		\end{minipage}%
		%\hfill
		\begin{minipage}[h]{0.34\linewidth}
			\center{\includegraphics[width=1.0\linewidth]{LagrangeHermit/Herm_Cheb_F3_N10.png}} 
		\end{minipage}%
		%\hfill
		\begin{minipage}[h]{0.34\linewidth}
			\center{\includegraphics[width=1.0\linewidth]{LagrangeHermit/Herm_Asin_F3_N10.png}} 
		\end{minipage}%
		\vfill
		\begin{minipage}[h]{0.34\linewidth}
			\center{\includegraphics[width=1.0\linewidth]{LagrangeHermit/Herm_Equi_F3_N20.png}} 
		\end{minipage}%
		%\hfill
		\begin{minipage}[h]{0.34\linewidth}
			\center{\includegraphics[width=1.0\linewidth]{LagrangeHermit/Herm_Cheb_F3_N20.png}} 
		\end{minipage}%
		%\hfill
		\begin{minipage}[h]{0.34\linewidth}
			\center{\includegraphics[width=1.0\linewidth]{LagrangeHermit/Herm_Asin_F3_N20.png}} 
		\end{minipage}%
		\vfill
		\begin{minipage}[h]{0.34\linewidth}
			\center{\includegraphics[width=1.0\linewidth]{LagrangeHermit/Herm_Equi_F3_N40.png}} 
		\end{minipage}%
		%\hfill
		\begin{minipage}[h]{0.34\linewidth}
			\center{\includegraphics[width=1.0\linewidth]{LagrangeHermit/Herm_Cheb_F3_N40.png}} 
		\end{minipage}%
		%\hfill
		\begin{minipage}[h]{0.34\linewidth}
			\center{\includegraphics[width=1.0\linewidth]{LagrangeHermit/Herm_Asin_F3_N40.png}} 
		\end{minipage}%
		\caption{Results of Hermit interpolation for 5, 10, 20 and 40 data points. The function is pictured with blue, its interpolant with red. First colomn corresponds to Equispaced data point distribution, second to Chebyshev and third to Asin.}
		%\label{ris:Area1-5}
	\end{figure}
 \newpage
	\subsection{Accuracy analysis}
	\begin{figure}[H]
		\begin{minipage}[h]{0.45\linewidth}
			\center{\includegraphics[width=1.0\linewidth]{LagrangeHermit/Lagr_eq_err_F3.png}}
			\caption{Dependence of error on the number of data\\ points for Lagrange interpolant and Equispaced point distribution.}
		\end{minipage}%
		\hspace{0.5cm}
		%\hfill
		\begin{minipage}[h]{0.45\linewidth}
			\center{\includegraphics[width=1.0\linewidth]{LagrangeHermit/Herm_eq_err_F3.png}} 
			\caption{Dependence of error on the number of data\\ points for Hermit interpolant and Equispaced point distribution.}
		\end{minipage}%
		\vfill
		\begin{minipage}[h]{0.45\linewidth}
			\center{\includegraphics[width=1.0\linewidth]{LagrangeHermit/Lagr_cheb_err_F3.png}} 
			\caption{Dependence of error on the number of data\\ points for Lagrange interpolant and Chebyshev point distribution.}
		\end{minipage}%
		\hspace{0.5cm}
		%\hfill
		\begin{minipage}[h]{0.45\linewidth}
			\center{\includegraphics[width=1.0\linewidth]{LagrangeHermit/Herm_cheb_err_F3.png}} 
			\caption{Dependence of error on the number of data\\ points for Hermit interpolant and Chebyshev point distribution.}
		\end{minipage}%
		%\hfill
		\vfill
		\begin{minipage}[h]{0.45\linewidth}
			\center{\includegraphics[width=1.0\linewidth]{LagrangeHermit/Lagr_asin_err_F3.png}} 
			\caption{Dependence of error on the number of data\\ points for Lagrange interpolant and Asin point distribution.}
		\end{minipage}%
		\hspace{0.5cm}
		%\hfill
		\begin{minipage}[h]{0.45\linewidth}
			\center{\includegraphics[width=1.0\linewidth]{LagrangeHermit/Herm_asin_err_F3.png}} 
			\caption{Dependence of error on the number of data\\ points for Hermit interpolant and Asin point distribution.}
		\end{minipage}%
		%\hfill
	\end{figure}
 New phenomenon can be seen: for $N=5$ Hermit interpolation using Equispaced distribution is actually more accurate than Chebyshev. This is most likely the consequence of non-smoothness at $x=\frac{1}{2}$.
	\newpage
	%%%%%%%%%%%%%%%%%%%%%%4
		\section{$  \sqrt{1-x^2}   $}
	\subsection{Lagrange interpolant}
	\begin{figure}[H]
		\begin{minipage}[h]{0.34\linewidth}
			\center{\includegraphics[width=1.0\linewidth]{LagrangeHermit/Lagr_Equi_F4_N10.png}}
		\end{minipage}%
		%\hfill
		\begin{minipage}[h]{0.34\linewidth}
			\center{\includegraphics[width=1.0\linewidth]{LagrangeHermit/Lagr_Cheb_F4_N10.png}}
		\end{minipage}%
		%\hfill
		\begin{minipage}[h]{0.34\linewidth}
			\center{\includegraphics[width=1.0\linewidth]{LagrangeHermit/Lagr_Asin_F4_N10.png}}
		\end{minipage}%
		\vfill
		\begin{minipage}[h]{0.34\linewidth}
			\center{\includegraphics[width=1.0\linewidth]{LagrangeHermit/Lagr_Equi_F4_N20.png}}
		\end{minipage}%
		%\hfill
		\begin{minipage}[h]{0.34\linewidth}
			\center{\includegraphics[width=1.0\linewidth]{LagrangeHermit/Lagr_Cheb_F4_N20.png}} 
		\end{minipage}%
		%\hfill
		\begin{minipage}[h]{0.34\linewidth}
			\center{\includegraphics[width=1.0\linewidth]{LagrangeHermit/Lagr_Asin_F4_N20.png}} 
		\end{minipage}%
		\vfill
		\begin{minipage}[h]{0.34\linewidth}
			\center{\includegraphics[width=1.0\linewidth]{LagrangeHermit/Lagr_Equi_F4_N40.png}} 
		\end{minipage}%
		%\hfill
		\begin{minipage}[h]{0.34\linewidth}
			\center{\includegraphics[width=1.0\linewidth]{LagrangeHermit/Lagr_Cheb_F4_N40.png}} 
		\end{minipage}%
		%\hfill
		\begin{minipage}[h]{0.34\linewidth}
			\center{\includegraphics[width=1.0\linewidth]{LagrangeHermit/Lagr_Asin_F4_N40.png}} 
		\end{minipage}%
		\vfill
		\begin{minipage}[h]{0.34\linewidth}
			\center{\includegraphics[width=1.0\linewidth]{LagrangeHermit/Lagr_Equi_F4_N80.png}} 
		\end{minipage}%
		%\hfill
		\begin{minipage}[h]{0.34\linewidth}
			\center{\includegraphics[width=1.0\linewidth]{LagrangeHermit/Lagr_Cheb_F4_N80.png}} 
		\end{minipage}%
		%\hfill
		\begin{minipage}[h]{0.34\linewidth}
			\center{\includegraphics[width=1.0\linewidth]{LagrangeHermit/Lagr_Asin_F4_N80.png}} 
		\end{minipage}%
		\caption{Results of Lagrange interpolation for 10, 20, 40 and 80 data points. The function is pictured with blue, its interpolant with red. First colomn corresponds to Equispaced data point distribution, second to Chebyshev and third to Asin.}
		%\label{ris:Area1-5}
	\end{figure}
	\newpage
	\subsection{Hermit interpolant}
	\begin{figure}[H]
		\begin{minipage}[h]{0.34\linewidth}
			\center{\includegraphics[width=1.0\linewidth]{LagrangeHermit/Herm_Equi_F4_N5.png}}
		\end{minipage}%
		%\hfill
		\begin{minipage}[h]{0.34\linewidth}
			\center{\includegraphics[width=1.0\linewidth]{LagrangeHermit/Herm_Cheb_F4_N5.png}}
		\end{minipage}%
		%\hfill
		\begin{minipage}[h]{0.34\linewidth}
			\center{\includegraphics[width=1.0\linewidth]{LagrangeHermit/Herm_Asin_F4_N5.png}}
		\end{minipage}%
		\vfill
		\begin{minipage}[h]{0.34\linewidth}
			\center{\includegraphics[width=1.0\linewidth]{LagrangeHermit/Herm_Equi_F4_N10.png}}
		\end{minipage}%
		%\hfill
		\begin{minipage}[h]{0.34\linewidth}
			\center{\includegraphics[width=1.0\linewidth]{LagrangeHermit/Herm_Cheb_F4_N10.png}} 
		\end{minipage}%
		%\hfill
		\begin{minipage}[h]{0.34\linewidth}
			\center{\includegraphics[width=1.0\linewidth]{LagrangeHermit/Herm_Asin_F4_N10.png}} 
		\end{minipage}%
		\vfill
		\begin{minipage}[h]{0.34\linewidth}
			\center{\includegraphics[width=1.0\linewidth]{LagrangeHermit/Herm_Equi_F4_N20.png}} 
		\end{minipage}%
		%\hfill
		\begin{minipage}[h]{0.34\linewidth}
			\center{\includegraphics[width=1.0\linewidth]{LagrangeHermit/Herm_Cheb_F4_N20.png}} 
		\end{minipage}%
		%\hfill
		\begin{minipage}[h]{0.34\linewidth}
			\center{\includegraphics[width=1.0\linewidth]{LagrangeHermit/Herm_Asin_F4_N20.png}} 
		\end{minipage}%
		\vfill
		\begin{minipage}[h]{0.34\linewidth}
			\center{\includegraphics[width=1.0\linewidth]{LagrangeHermit/Herm_Equi_F4_N40.png}} 
		\end{minipage}%
		%\hfill
		\begin{minipage}[h]{0.34\linewidth}
			\center{\includegraphics[width=1.0\linewidth]{LagrangeHermit/Herm_Cheb_F4_N40.png}} 
		\end{minipage}%
		%\hfill
		\begin{minipage}[h]{0.34\linewidth}
			\center{\includegraphics[width=1.0\linewidth]{LagrangeHermit/Herm_Asin_F4_N40.png}} 
		\end{minipage}%
		\caption{Results of Hermit interpolation for 5, 10, 20 and 40 data points. The function is pictured with blue, its interpolant with red. First colomn corresponds to Equispaced data point distribution, second to Chebyshev and third to Asin.}
		%\label{ris:Area1-5}
	\end{figure}
	\newpage
	\subsection{Accuracy analysis}
	\begin{figure}[H]
		\begin{minipage}[h]{0.45\linewidth}
			\center{\includegraphics[width=1.0\linewidth]{LagrangeHermit/Lagr_eq_err_F4.png}}
			\caption{Dependence of error on the number of data\\ points for Lagrange interpolant and Equispaced point distribution.}
		\end{minipage}%
		\hspace{0.5cm}
		%\hfill
		\begin{minipage}[h]{0.45\linewidth}
			\center{\includegraphics[width=1.0\linewidth]{LagrangeHermit/Herm_eq_err_F4.png}} 
			\caption{Dependence of error on the number of data\\ points for Hermit interpolant and Equispaced point distribution.}
		\end{minipage}%
		\vfill
		\begin{minipage}[h]{0.45\linewidth}
			\center{\includegraphics[width=1.0\linewidth]{LagrangeHermit/Lagr_cheb_err_F4.png}} 
			\caption{Dependence of error on the number of data\\ points for Lagrange interpolant and Chebyshev point distribution.}
		\end{minipage}%
		\hspace{0.5cm}
		%\hfill
		\begin{minipage}[h]{0.45\linewidth}
			\center{\includegraphics[width=1.0\linewidth]{LagrangeHermit/Herm_cheb_err_F4.png}} 
			\caption{Dependence of error on the number of data\\ points for Hermit interpolant and Chebyshev point distribution.}
		\end{minipage}%
		%\hfill
		\vfill
		\begin{minipage}[h]{0.45\linewidth}
			\center{\includegraphics[width=1.0\linewidth]{LagrangeHermit/Lagr_asin_err_F4.png}} 
			\caption{Dependence of error on the number of data\\ points for Lagrange interpolant and Asin point distribution.}
		\end{minipage}%
		\hspace{0.5cm}
		%\hfill
		\begin{minipage}[h]{0.45\linewidth}
			\center{\includegraphics[width=1.0\linewidth]{LagrangeHermit/Herm_asin_err_F4.png}} 
			\caption{Dependence of error on the number of data\\ points for Hermit interpolant and Asin point distribution.}
		\end{minipage}%
		%\hfill
	\end{figure}
Mostly the same dependencies can be seen for this function. However, it should be noted that for Hermit interpolation the rightmost point was mooved from $x=1$ to $x=1.0001$, because the first derivative at $x=1$ is infinite.
	\newpage
	\part{Cubic spline interpolation}
	\section{Parametrization}
	Cubic spline interpolation of an ellipse:
	\begin{equation}
			\label{ellipse}		
			x^2 + \frac{y^2}{2}=1,
	\end{equation}
 is considered. Since the curve satisfying Eq.(\ref{ellipse}) can not be expressed in a form $y(x)$, we will work with its parametrization $(x(t),y(t))$. A set of data points is generated from:
	\begin{equation}
		\begin{cases}\ds
			\label{ell_param}		
			x=cos(t),\\
			y=\sqrt{2}sin(t),
		\end{cases}
	\end{equation}
	where $t \in [0,2\pi+\delta]$. The interpolation was performed for $N=$9, 13, 17, and 21 data points, extension of the interval $\delta = \frac{2\pi}{N-1}$ is introduced to apply periodic boundary conditions: $f''(N-1)=f(0), \quad f''(N)=f(1) $. 
	\section{Results}
	\begin{figure}[H]
		\begin{minipage}[h]{0.5\linewidth}
		\center{\includegraphics[width=1.0\linewidth]{CubicSpline/cubic_spline_N8.png}} \\
		\caption{Interpolant for $N=9$.}
		\end{minipage}%
		\hspace{0.5cm}
		%\hfill
		\begin{minipage}[h]{0.5\linewidth}
			\center{\includegraphics[width=1.0\linewidth]{CubicSpline/cubic_spline_N12.png}} \\
		\caption{Interpolant for $N=13$.}
		\end{minipage}%
		\vfill
		\begin{minipage}[h]{0.5\linewidth}
			\center{\includegraphics[width=1.0\linewidth]{CubicSpline/cubic_spline_N16.png}} 
		\caption{Interpolant for $N=17$.}
		\end{minipage}%
		\hspace{0.5cm}
		%\hfill
		\begin{minipage}[h]{0.5\linewidth}
				\center{\includegraphics[width=1.0\linewidth]{CubicSpline/cubic_spline_N20.png}} 
			\caption{Interpolant for $N=21$.}
		\end{minipage}%
	\caption{Cubic spline interpolant is pictured with red and the actual function with blue.}
	\end{figure}

	\part{Finite difference and Pad\'e approximation}
	\section{Finite difference}
	The most accurate finite difference formula for $f''(x_{i})$ of a function $f(x)$ known at points $x_{i-1}$, $x_{i}$, $x_{i+1}$, $x_{i+2}$, and $x_{i+3}$ is considered. The points are equispaced with distance $h$ between them. The problem can be reduced to finding coefficients $a$, $b$, $c$, $d$, and $e$ such that 
	\begin{equation}
		\label{fin_dif}
		f''(x_{i})= af(x_{i-1})+bf(x_{i})+cf(x_{i+1})+df(x_{i+2})+ef(x_{i+3}) + Ch^p + O(h^{p+1})
		\end{equation}
			 with maximum possible $p$.
	By substituting each function with its Taylor series expansion around $x_{i}$ and requiring the coefficient in front of second derivative to be qual 1 and all other 0 the following equations for $a$, $b$, $c$, $d$, and $e$ can be obtained:
	\begin{equation}
		\begin{cases}\ds
			\label{d2_system}		
			a+b+c+d+e=0,\\
			- h \cdot a + h\cdot c+2\cdot h\cdot d+3\cdot h \cdot e=0,\\
			\frac{h^2}{2} \cdot a + \frac{h^2}{2}\cdot c+2\cdot h^2\cdot d+\frac{9 \cdot h^2}{2} \cdot e=1,\\
			- \frac{h^3}{6} \cdot a + \frac{h^3}{6}\cdot c+\frac{4 \cdot h^3}{3}\cdot d+ \frac{9 \cdot h^3}{2} \cdot e=0,\\
			\frac{h^4}{24} \cdot a + \frac{h^4}{24}\cdot c+\frac{2 \cdot h^4}{3}\cdot d+ \frac{27 \cdot h^4}{8} \cdot e=0,\\
		\end{cases}
	\end{equation}
	By substituting the solution of System(\ref{d2_system}) into Eq.(\ref{fin_dif}) we obtain:
	\begin{equation}
	f''_i= \frac{11f_{i-1}-20f_{i}+6f_{i+1}+4f_{i+2}-f_{i+3}}{12h^2} + \frac{h^3}{12} +  O(h^4)
	\end{equation}
		
	\section{Pad\'e approximation}
	
	The most accurate Pad\'e approximation for $f'(x_{i})$ of a function $f(x)$ known at points $x_{i-2}$, $x_{i-1}$, and $x_{i}$ is considered. The points are equispaced with distance $h$ between them. The problem can be reduced to finding coefficients $a$, $b$, $c$, $d$, and $e$ such that 
	\begin{equation}
		\label{pade_dif}
		h(af'(x_{i-2})+bf'(x_{i-1})+f'(x_{i}))= cf(x_{i-2})+df(x_{i-1})+ef(x_{i}) + Ch^p + O(h^{p+1})
	\end{equation}
	with maximum possible $p$.
	By substituting each function with its Taylor series expansion around $x_{i}$ and requiring the coefficient in front of each derivative from the right side to be qual to the coefficient in front of the same derivative from the left side following equations for $a$, $b$, $c$, $d$, and $e$ can be obtained:
	\begin{equation}
		\begin{cases}\ds
			\label{d1_system}		
			- c - d - e == 0,\\
			2\cdot c \cdot h + d \cdot h + h \cdot(a + b + 1)=0,\\
			 - h\cdot(2\cdot a\cdot h + b\cdot h) - 2\cdot c\cdot h^2 - \dfrac{d\cdot h^2}{2}=0,\\
			h\cdot(2\cdot a\cdot h^2 + \frac{b\cdot h^2}{2}) + \frac{4\cdot c\cdot h^3}{3} + \frac{d\cdot h^3}{6}=0,\\
			- h\cdot (\frac{4\cdot a\cdot h^3}{3} + \frac{b\cdot h^3}{6}) - \frac{2\cdot c\cdot h^4}{3} - \frac{d\cdot h^4}{24}=0,\\
		\end{cases}
	\end{equation}
	By substituting the solution of System(\ref{d1_system}) into Eq.(\ref{pade_dif}) we obtain:
	\begin{equation}
		h(f'(x_{i-2})+4f'(x_{i-1})+f'(x_{i}))= -3f(x_{i-2})+3f(x_{i}) + \frac{h^5}{30} + O(h^{6})
	\end{equation}
	\newpage
	\part{Numeric integration}
	Different formulas for numeric integraion of
	\begin{equation}
		\int_0^1 (\dfrac{1}{(x-1)^2+0.002} + \dfrac{1}{(x-0.2)^2+0.005} -5)\cdot dx = \frac{atan(\frac{1}{\sqrt{0.002}})}{\sqrt{0.002}}+\frac{atan(\frac{0.8}{\sqrt{0.005}})}{\sqrt{0.005}}+\frac{atan(\frac{0.2}{\sqrt{0.005}})}{\sqrt{0.005}}-5
	\end{equation}
are considered. In each method except for adaptive quadrature the interval is divided into $N$ equal segments of length $h=\frac{1}{N}$ and results are obtained for $N=$8, 16, 32, 64, 128, 256.
	\section{Trapezoidal Rule}
	$$
	I\approx h\left(\frac{f_0+f_N}{2} + \sum_{j=1}^{N-1}f_j\right)
	$$ 
	\begin{figure}[H]
		\center{\includegraphics[scale=0.8]{Integrals/Trapesoidal.png}} \\
		\caption{Trapesoidal Rule. Error vs N is pictured with blue and approximation of order of accuracy with red.}
	\end{figure}
The results show that the method is at least second order accurate for the considered integral.
	\section{Simpson's Rule}
	$$
	I\approx\frac{h}{3}\left(f_0+f_N +4\cdot \sum_{\begin{array}{l}
			~~j=1 \\ 
			j=odd
	\end{array}}^{N-1}f_j + 2\cdot\sum_{\begin{array}{l}
			~~j=1 \\ 
			j=even
		\end{array}}^{N-1}f_j\right)
	$$
	\begin{figure}[H]
		\center{\includegraphics[scale=0.8]{Integrals/Simpson.png}} \\
		\caption{Simpson's Rule. Error vs N is pictured with blue and approximation of order of accuracy with red.}
	\end{figure}
The results show that the method is even higher than third order accurate for the considered integral.
	\section{Trapezoidal Rule with End-Correction}
	$$
	I\approx h\left(\frac{f_0+f_N}{2} + \sum_{j=1}^{N-1}f_j\right) - \frac{h^2}{12}(f'_N - f'_0)
	$$ 
	\begin{figure}[H]
		\center{\includegraphics[scale=0.8]{Integrals/Trapesoidal_End.png}} \\
		\caption{Trapezoidal with End=Correction Rule. Error vs N is pictured with blue and approximation of order of accuracy with red.}
	\end{figure}
	The results show that the method is even higher than forth order accurate for the considered integral.
	\newpage
	\section{Adaptive Quadrature}
	The approximation of the integral over a segment $[a,b]$ is obtained using Rectangular Rule: $\int_a^b f(x)~dx \approx (b-a)\cdot f(\frac{a+b}{2}) = I_{[a,b]} $. The error of approximation is estimated using Richardson Extrapolation. If the value of $ |I_{[a,b]} - (I_{[a,\frac{a+b}{2}]}+I_{[\frac{a+b}{2},b]})|$ is less than some predetermined tolerance, $tol$, $I_{[a,b]}$ is taken as the approximation, else the same prosedure is performed for $(I_{[a,\frac{a+b}{2}]}$ and $I_{[\frac{a+b}{2},b]})$ with $\widetilde{tol}=\frac{tol}{2}$, the point $x=\frac{a+b}{2}$ is added to the resulting distribution of dividing points. The recursive procedure is started with $[a,b] = [0,1]$. Numerical calculations were conducted for $tol=10^{-1},~10^{-3},~10^{-5},~10^{-7}$.
	\begin{figure}[H]
		\begin{minipage}[h]{0.5\linewidth}
			\center{\includegraphics[width=1.0\linewidth]{Integrals/Adaptive_1.png}} \\
			\caption{The function is pictured with blue line and point distribution for $tol=10^{-1}$ with red circles.}
		\end{minipage}%
		\hspace{0.5cm}
		%\hfill
		\begin{minipage}[h]{0.5\linewidth}
			\center{\includegraphics[width=1.0\linewidth]{Integrals/Adaptive_2.png}} \\
			\caption{The function is pictured with blue line and point distribution for $tol=10^{-3}$ with red circles.}
		\end{minipage}%
		\vfill
		\begin{minipage}[h]{0.5\linewidth}
			\center{\includegraphics[width=1.0\linewidth]{Integrals/Adaptive_3.png}} 
			\caption{The function is pictured with blue line and point distribution for $tol=10^{-5}$ with red circles.}
		\end{minipage}%
		\hspace{0.5cm}
		%\hfill
		\begin{minipage}[h]{0.5\linewidth}
			\center{\includegraphics[width=1.0\linewidth]{Integrals/Adaptive_4.png}} 
			\caption{The function is pictured with blue line and point distribution for $tol=10^{-7}$ with red circles.}
		\end{minipage}%
	\end{figure}
The leading error term for Rectangular Rule depends on the second derivative of the function. It is natural that the distribution of points for adaptive quadrature follows it.
\begin{figure}[H]
	\center{\includegraphics[scale=0.8]{Integrals/Adaptive_distrib.png}} \\
	\caption{Histogram with 100 equispaced bins is pictured with blue and $|f''(x)|\cdot 10^{-2}$ with red.}
\end{figure}
\newpage
	\part{Numeric integration of improper integrals}
	\section{Semi-Infinite intervals}
	The following formula for numerical integration of improper integrals is considered:
	$\int_0^\infty e^{-x}f(x)~ dx \approx \sum_{j=0}^{N}\omega_{j}f(x_j)$, where $x_j$ and $\omega_{j}$ are zeros and weight factors of Laguerre polinomial corresponding to the chosen number of points $N$. Numerical integration for $N=$2, 3, 4, 5 will be conducted.
	\subsection{$\int_0^\infty e^{-10x}sin(x)~ dx$}
	\begin{equation}
		I = \int_0^\infty e^{-10x}sin(x)~ dx = \frac{1}{10}\int_0^\infty e^{-10x}cos(x)~ dx = \frac{1}{100} - \frac{1}{100}\int_0^\infty e^{-10x}sin(x)~ dx => I= \frac{1}{101},
	\end{equation}
	\begin{equation}
		\int_0^\infty e^{-10x}sin(x)~ dx = \frac{1}{10}\int_0^\infty e^{-t}sin(\frac{t}{10})~ dt \approx \sum_{j=0}^{N}\omega_{j}sin(\frac{x_j}{10}).
	\end{equation}
	
	\begin{figure}[H]
		\center{\includegraphics[scale=0.8]{Integrals/Improper1.png}} \\
		\caption{ Error vs N is pictured with blue and approximation of order of accuracy with red.}
	\end{figure}
\subsection{$\int_0^\infty \frac{e^{-x}}{1+e^{-2x}}~ dx$}
\begin{equation}
	I = \int_0^\infty \frac{e^{-x}}{1+e^{-2x}}~ dx=-\int_0^\infty \frac{1}{1+e^{-2x}}~ de^{-x} =atan(1) = \frac{\pi}{4},
\end{equation}
\begin{equation}
	\int_0^\infty \frac{e^{-x}}{1+e^{-2x}}~ dx \approx \sum_{j=0}^{N}\omega_{j}\frac{1}{1+e^{-2x_j}}.
\end{equation}
\begin{figure}[H]
	\center{\includegraphics[scale=0.8]{Integrals/Improper2.png}} \\
	\caption{ Error vs N is pictured with blue and approximation of order of accuracy with red.}
\end{figure}
	\section{Infinite intervals}
	The following formula for numerical integration of improper integrals is considered:
	$\int_{-\infty}^\infty e^{-x^2}f(x)~ dx \approx \sum_{j=0}^{N}\omega_{j}f(x_j)$, where $x_j$ and $\omega_{j}$ are zeros and weight factors of Hermit polinomial corresponding to the chosen number of points $N$. Numerical integration for $N=$2,3,4,5 will be conducted.
	\subsection{$\int_{-\infty}^\infty |x|e^{-3x^2}~ dx$}
	\begin{equation}
		I = \int_{-\infty}^\infty |x|e^{-3x^2}~ dx =\int_0^\infty e^{-3x^2}~ dx^2 =\frac{1}{3},
	\end{equation}
	\begin{equation}
		\int_{-\infty}^\infty |x|e^{-3x^2}~ dx = \frac{1}{3}\int_{-\infty}^\infty |t|e^{-t^2}~ dt \approx \sum_{j=0}^{N}\omega_{j}\frac{1}{3}|x_j|.
	\end{equation}
	\begin{figure}[H]
		\center{\includegraphics[scale=0.8]{Integrals/Improper3.png}} \\
		\caption{ Error vs N is pictured with blue and approximation of order of accuracy with red.}
	\end{figure}
	\subsection{$\int_{-\infty}^\infty e^{-x^2}cos(x)~ dx$}
\begin{equation}
	I(\alpha) = \int_{-\infty}^\infty  e^{-x^2}cos(\alpha x)~ dx,\quad I(0)=\int_{-\infty}^\infty  e^{-x^2}~ dx=\sqrt{\pi}
\end{equation}
\begin{equation}
	I'(\alpha) = -\int_{-\infty}^\infty  xe^{-x^2}sin(\alpha x)~ dx= -\frac{\alpha}{2}\int_{-\infty}^\infty  e^{-x^2}cos(\alpha x)~ dx = -\frac{\alpha}{2} I(\alpha),
\end{equation}
$$
	I'(\alpha) = -\frac{\alpha}{2} I(\alpha) \quad => \quad I(\alpha)=I(0)e^{\frac{-\alpha^2}{4}} \quad => \quad \int_{-\infty}^\infty e^{-x^2}cos(x)~ dx = I(1) = \frac{\sqrt{\pi}}{e^{\frac{1}{4}}}
$$
\begin{equation}
	\int_{-\infty}^\infty e^{-x^2}cos(x)~ dx  \approx \sum_{j=0}^{N}\omega_{j}cos(x_j).
\end{equation}
\begin{figure}[H]
	\center{\includegraphics[scale=0.8]{Integrals/Improper4.png}} \\
	\caption{ Error vs N is pictured with blue and approximation of order of accuracy with red.}
\end{figure}
\end{document}